% Options for packages loaded elsewhere
\PassOptionsToPackage{unicode}{hyperref}
\PassOptionsToPackage{hyphens}{url}
%
\documentclass[
]{article}
\usepackage{amsmath,amssymb}
\usepackage{lmodern}
\usepackage{iftex}
\ifPDFTeX
  \usepackage[T1]{fontenc}
  \usepackage[utf8]{inputenc}
  \usepackage{textcomp} % provide euro and other symbols
\else % if luatex or xetex
  \usepackage{unicode-math}
  \defaultfontfeatures{Scale=MatchLowercase}
  \defaultfontfeatures[\rmfamily]{Ligatures=TeX,Scale=1}
\fi
% Use upquote if available, for straight quotes in verbatim environments
\IfFileExists{upquote.sty}{\usepackage{upquote}}{}
\IfFileExists{microtype.sty}{% use microtype if available
  \usepackage[]{microtype}
  \UseMicrotypeSet[protrusion]{basicmath} % disable protrusion for tt fonts
}{}
\makeatletter
\@ifundefined{KOMAClassName}{% if non-KOMA class
  \IfFileExists{parskip.sty}{%
    \usepackage{parskip}
  }{% else
    \setlength{\parindent}{0pt}
    \setlength{\parskip}{6pt plus 2pt minus 1pt}}
}{% if KOMA class
  \KOMAoptions{parskip=half}}
\makeatother
\usepackage{xcolor}
\usepackage[margin=1in]{geometry}
\usepackage{graphicx}
\makeatletter
\def\maxwidth{\ifdim\Gin@nat@width>\linewidth\linewidth\else\Gin@nat@width\fi}
\def\maxheight{\ifdim\Gin@nat@height>\textheight\textheight\else\Gin@nat@height\fi}
\makeatother
% Scale images if necessary, so that they will not overflow the page
% margins by default, and it is still possible to overwrite the defaults
% using explicit options in \includegraphics[width, height, ...]{}
\setkeys{Gin}{width=\maxwidth,height=\maxheight,keepaspectratio}
% Set default figure placement to htbp
\makeatletter
\def\fps@figure{htbp}
\makeatother
\setlength{\emergencystretch}{3em} % prevent overfull lines
\providecommand{\tightlist}{%
  \setlength{\itemsep}{0pt}\setlength{\parskip}{0pt}}
\setcounter{secnumdepth}{-\maxdimen} % remove section numbering
\ifLuaTeX
  \usepackage{selnolig}  % disable illegal ligatures
\fi
\IfFileExists{bookmark.sty}{\usepackage{bookmark}}{\usepackage{hyperref}}
\IfFileExists{xurl.sty}{\usepackage{xurl}}{} % add URL line breaks if available
\urlstyle{same} % disable monospaced font for URLs
\hypersetup{
  pdftitle={Mote Marine Laboratory TagLab Analysis Standard Operating Procedure},
  pdfauthor={Written by: Gabrielle D'Alonzo (gdalonzo@mote.org)},
  hidelinks,
  pdfcreator={LaTeX via pandoc}}

\title{Mote Marine Laboratory TagLab Analysis Standard Operating
Procedure}
\author{Written by: Gabrielle D'Alonzo
(\href{mailto:gdalonzo@mote.org}{\nolinkurl{gdalonzo@mote.org}})}
\date{}

\begin{document}
\maketitle

\hypertarget{last-updated-gabrielle-dalonzo-october-11-2024}{%
\subsubsection{Last Updated: Gabrielle D'Alonzo, October 11,
2024}\label{last-updated-gabrielle-dalonzo-october-11-2024}}

\hypertarget{overview}{%
\subsection{Overview}\label{overview}}

Contained in this document is the Standard Operating Procedure for the
analysis of large orthophotos/orthomosaics using the open-source
software TagLab. The preexisting documentation for TagLab that can be
found on their \href{https://taglab.isti.cnr.it/}{website} and
\href{https://github.com/cnr-isti-vclab/TagLab}{GitHub} are great
resources that lay out the basic
\href{https://github.com/cnr-isti-vclab/TagLab/wiki/Install-TagLab}{installation},
and \href{https://taglab.isti.cnr.it/docs}{tools} to successfully
navigate TagLab. These resources also include some potentially relevant
\href{https://github.com/cnr-isti-vclab/TagLab/wiki/Install-TagLab}{troubleshooting
tips}. Any confusion on program usage not addressed in this protocol may
be resolved there. Additional Standard Operating Procedures for other
aspects within the photogrammetric pipeline can be found
\href{https://github.com/icombs2017/MoteSOPs}{here}. The NOAA fork (new
version) of Taglab can be accessed through this github link for any
information regarding new features and installation:
\url{https://github.com/Jordan-Pierce/TagLab?tab=readme-ov-file}

\hypertarget{project-set-up}{%
\section{Project Set-Up}\label{project-set-up}}

\begin{center}\rule{0.5\linewidth}{0.5pt}\end{center}

\begin{enumerate}
\def\labelenumi{\arabic{enumi}.}
\item
  Open TagLab using either the desktop shortcut or the TagLab.py file
  located in your local GitHub/TagLab folder.
\item
  To load an image, select Project → Add New Map.
\item
  Fill out the relevant fields

  \begin{itemize}
  \tightlist
  \item
    Map Name: What you want to call that specific image

    \begin{itemize}
    \tightlist
    \item
      Mote uses the following naming convention
      YYYY\_MM\_DD\_Site-Name.fileFormat ex: 2022\_09\_21\_Q-9.tiff
    \end{itemize}
  \item
    RGB Image: either type in the file path to your image or click the
    symbol to the right ``\ldots{}'' and navigate via your file browser
  \item
    Depth Image: do the same thing if you have a Digital Elevation Model
    (DEM) coregistered to your image
  \item
    Acquisition Date: Fill in the date for when the imagery was captured
    format: YYYY-MM-DD
  \item
    Pixel size (mm): How long one pixel is (mm), this can also be done
    post-hoc using the measure tool
  \end{itemize}
\end{enumerate}

\textbf{Note:} \emph{Map Name}, \emph{Path to RGB Image}, and
\emph{Acquisition Date} are necessary in order to start your project.

\begin{enumerate}
\def\labelenumi{\arabic{enumi}.}
\setcounter{enumi}{3}
\item
  Click Apply.
\item
  Repeat this for every image you wish to analyze in your particular
  project
\end{enumerate}

\textbf{Note:} Our projects tend to include a single site over multiple
time points and we add subsequent time points to the project as they
enter the analysis pipelines.

\begin{enumerate}
\def\labelenumi{\arabic{enumi}.}
\setcounter{enumi}{5}
\tightlist
\item
  Save the project by navigating to File → Save As and then save the
  project within the ``TagLab'' folder of whatever project you are
  working on
\end{enumerate}

\textbf{Note:} This is Mote specific, but it is wise to save your TagLab
Project in the same folder as whatever images are associated with it, if
there are multiple images within multiple folders, make sure the TagLab
Project is a level above all of the folders that contain images for that
specific project.

\begin{enumerate}
\def\labelenumi{\arabic{enumi}.}
\setcounter{enumi}{6}
\tightlist
\item
  You should now be able to see your imported image on screen and are
  ready to begin analyzing your image.
\end{enumerate}

\#Opening NOAA Taglab Fork ***

\begin{enumerate}
\def\labelenumi{\arabic{enumi}.}
\item
  Open the Command or Anaconda prompt on your system.
\item
  Enter the following comands one at a time:
\end{enumerate}

\begin{itemize}
\tightlist
\item
  conda activate Taglab -- changing from base directory to Taglab *cd
  ``your/file/path/to/Taglab/folder/goes/here'' -- changing working
  directory to correct file path
\item
  python Taglab.py -- opening Taglab
\end{itemize}

Note: This new version of Taglab does not work on MacOS as of yet.

Reopening Working Files ***

\begin{enumerate}
\def\labelenumi{\arabic{enumi}.}
\item
  Recently accessed files may appear when File drop-down is clicked.
\item
  If files do not appear in drop-down, click Open Project (shortcut Ctrl
  + O), also in File drop down.
\item
  Navigate to the .JSON file you desire to open, and select.
\item
  If file paths have been changed since last access, Taglab will present
  a prompt to re-establish those file paths. Follow the instruction of
  the prompts and select the Ortho-images it seeks. File names will be
  given by Taglab to make it clear which time point you should be
  selecting.
\end{enumerate}

\hypertarget{image-analysis}{%
\section{Image Analysis}\label{image-analysis}}

\begin{center}\rule{0.5\linewidth}{0.5pt}\end{center}

\begin{enumerate}
\def\labelenumi{\arabic{enumi}.}
\tightlist
\item
  To set the dictionary, navigate to Project → Labels Dictionary
  Editor\ldots{} and select Load.

  \begin{itemize}
  \tightlist
  \item
    Navigate to where you are storing your dictionary
  \item
    Alternatively, you can build your own

    \begin{itemize}
    \tightlist
    \item
      Select an RGB color
    \item
      Add the label name
    \item
      select ``Add''
    \item
      Repeat this for as many entries as you need, alternatively, this
      can be done piecemeal while you segment
    \end{itemize}
  \end{itemize}
\item
  Since scale was not set at the inception of the project, before we
  begin segmenting we must set the scale. Navigate to the Measure Tool,
  ninth on the list in the left hand tool bar (Shortcut: 9)

  \begin{itemize}
  \tightlist
  \item
    If you know your mm/pixel measurements you can input that manually
  \item
    If not, select ``Set new pixel size''
  \item
    Use your cursor to select two ends of a known distance
  \item
    Enter your known distance (in mm) and select OK
  \item
    Measure that known distance, or another known distance using the
    same tool to double check that your scale was set correctly
  \end{itemize}
\end{enumerate}

\textbf{Note:} We use scale bars with their centers spaced 50cm apart

\begin{enumerate}
\def\labelenumi{\arabic{enumi}.}
\setcounter{enumi}{2}
\tightlist
\item
  To set your working area navigate to Project → Set Working Area

  \begin{itemize}
  \tightlist
  \item
    Select the square with a cursor (to the left of the ``Set'' button)
  \item
    Drag the square to encompass your working area
  \end{itemize}
\end{enumerate}

\textbf{Note:} We delineate our working areas using markers/pins

\textbf{Note:} You can enter your working area manually using local
coordinates, but this is not recommended.

\begin{enumerate}
\def\labelenumi{\arabic{enumi}.}
\setcounter{enumi}{3}
\tightlist
\item
  You can then use the Create Grid tool to make a grid to place over the
  working area. This breaks the whole ortho down into more manageable
  pieces for more thorough segmentation.

  \begin{itemize}
  \tightlist
  \item
    In addition, Active/Deactivate Grid Operations allows for boxes of
    the grid to be marked as complete, incomplete, or empty to help
    track segmentation progress through the whole ortho.
  \end{itemize}
\item
  Begin segmenting your image

  \begin{itemize}
  \tightlist
  \item
    A list of the tools and functions can be found
    \href{https://taglab.isti.cnr.it/docs}{here}
  \item
    We tend to use the Positive/negative clicks segmentation tool, the
    4-clicks segmentation tool, and the Edit Border tool the most (in
    that order)
  \end{itemize}
\item
  Once the image is segmented, navigate to File → Export and export your
  data in your preferred format
\end{enumerate}

\textbf{Note:} We normally export as a Data Table for downstream
statistical analysis and data visualization in R. However, you might
find other formats more or less useful depending on your research
questions.

\end{document}
